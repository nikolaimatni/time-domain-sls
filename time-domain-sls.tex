\documentclass[11pt]{article}

\usepackage{graphicx, amsmath,amsthm,amssymb,url}
\usepackage{empheq}

\usepackage[colorlinks=true,linkcolor=blue,urlcolor=black,citecolor=blue,breaklinks]{hyperref}
\usepackage[labelfont=bf]{subcaption}
\usepackage[labelfont=bf]{caption}
\usepackage{algorithm,algcompatible,mathtools}
\usepackage[multidot]{grffile}
\usepackage[normalem]{ulem} % just for strikeout
\usepackage{enumerate}
\usepackage{thmtools}
\usepackage{thm-restate}

\usepackage{fullpage,etoolbox}
\usepackage{color}
\usepackage[draft]{commands}

\newcommand*\widefbox[1]{\fbox{\hspace{1em}#1\hspace{1em}}}
\numberwithin{equation}{section}
\def \basefigwidth{0.4}
% Fix space breaking
\relpenalty=9999
\binoppenalty=9999


\title{Working Note: Time-Domain System Level Synthesis}
\author{Nikolai Matni}
\date{\today}

\begin{document}
\maketitle

\section{A time-domain derivation}
Consider the system
\begin{equation}
x_{k+1} = Ax_k + Bu_k + w_k.
\label{eq:dynamics}
\end{equation}
In what follows, we will use $z(t)$ to denote the signal $z$ over time, and $z_k$ to denote its instantaneous value at $t=k$, i.e., $z(t) = (z_0,z_1,\dots)$.  For a matrix $X$, we use $X^j$ to denote its $j$th column; we also sometimes overload notation and use a superscript as a label for a vector, i.e., we will sometimes write $x^j(t)$ to denote a specific instantiation of a generic signal $x$.

Our goal is to characterize the closed loop maps from $w(t) \to x(t)$ and $w(t) \to u(t)$ under the assumption that $u(t) = (K\star x)(t)$ for some LTI controller $K(t)$.  Here $(g\star h)(t)$ denotes the convolution of signals $g(t)$ and $h(t)$.  Specifically, we seek filters $R(t)$ and $M(t)$ such that
\begin{equation}
\begin{array}{rcl}
x(t) &=& (R\star w)(t) \\
u(t) &=& (M\star w)(t),
\end{array}
\label{eq:responses}
\end{equation}
under the assumption that $u(t) = (K\star x)(t)$.

First note that for any disturbance signal $w(t)$ we can write $w(t) = \sum_k \sum_j w_{jk}e_j\delta_{k-t}$, and hence by linearity of our dynamics \eqref{eq:dynamics}, it suffices to characterize the impulse responses of the system due to $w^j(t):= e_j \delta_k$.  To that end, we write $x^j(t)$ and $u^j(t)$ to denote the state and control responses due to the $j$th impulse disturbance $w^j(t)$; the dynamics \eqref{eq:dynamics} then simplify to
\begin{equation}
R^j_{k+1} = AR^j_k + BM^j_k, \ R^j_1 = e_j,
\end{equation}
where we have used the fact that the resulting state and control responses satisfy, by definition,
\[
\begin{array}{rclclcl}
x^j_k &=& \sum_{t=1}^k R_tw^j_{k-t} &=& \sum_{t=1}^k R_te_j \delta_{k-t} &=& R^j_k \\
u^j_k &=& \sum_{t=1}^k M_tw^j_{k-t} &=& \sum_{t=1}^k M_te_j \delta_{k-t} &=& M^j_k.
\end{array}
\]

Therefore, concatenating the respective responses, we have that the responses $R(t)$ and $M(t)$ must satsify
\begin{equation}
R_{k+1} = AR_k + BM_k, \ R_1 = I.
\label{eq:constraints}
\end{equation}

Our next claim is that since $R(t)$ and $M(t)$ satisfy $R_0 = 0$ and $M_0=0$, the controller can be implemented directly as $u(t) = (M\star w)(t)$ by noting that $u_k$ only depends on $w_1,\dots,w_{k-1}$, and that at time $k$, one can compute $w_{k-1} = x_k - (Ax_{k-1} + Bu_{k-1})$.  It therefore follows from the constraints \eqref{eq:constraints} that the desired state response is also achieved.  To make this explicit, notice that for $w(t) = w^j(t) = e_j \delta_k$ we have from the dynamics \eqref{eq:dynamics} that
\[
x^j_1 = e_j = R^j_1.
\]
Similarly, we have that
\[
x^j_2 = Ax^j_1 + Bu^j_1 = AR^j_1 + BM^j_1 = R^j_2
\]
where the final equality follows from the constraints \eqref{eq:constraints}.  This argument is easily extended to any $k\geq 2$ by induction.

Note that this argument is valid over finite or infinite horizons -- note that in the infinite horizon case, additional stability constraints must be imposed on $R(t)$ and $M(t)$ to ensure that the resulting closed loop system is itself stable.

\section{Finite-horizon LQR}
\subsection{Zero IC, Gaussian Noise}
Here we consider the finite-horizon LQR problem with zero initial condition, driven by Gaussian noise:
\begin{equation}
\begin{array}{rl}
\min_{x(t),u(t)} & \sum_{k=1}^T \E\left[x_k^\tp Q x_k + u_k^\tp Su_k\right] + \E\left[x_{T+1}^\tp Q_F x_{T+1}\right] \\
\st & x_{k+1} = Ax_k + Bu_k + w_k, \ x_0 = 0,
\end{array}
\label{eq:lqr_noise}
\end{equation}
for $w_k {\iid}\mathcal{N}(0,I)$ (note that this is wlog assuming a non-degenerate distribution for the process noise).  We now show how this problem can be recast in terms of the system responses $R(t)$ and $M(t)$.  First note that given equations \eqref{eq:responses}, it follows that
\begin{equation}
\begin{array}{rcl}
x_k &=& \sum_{t=1}^k R_t w_{k-t} \\
u_k &=& \sum_{t=1}^k M_t w_{k-t}.
\end{array}
\label{eq:dist}
\end{equation}
Exploiting the i.i.d. nature of the disturbance, we can then write
\begin{equation}
\begin{array}{rcl}
\E\left[x_k^\tp Q x_k\right] &=& \sum_{t=1}^k \Tr R_t^\tp Q R_t \\
\E\left[u_k^\tp S u_k\right] &=& \sum_{t=1}^k \Tr M_t^\tp S M_t \\
\E\left[x_{T+1}^\tp Q_F x_{T+1}\right] &=& \sum_{t=1}^{T+1} \Tr R_t^\tp Q_F R_t.
\end{array}
\label{eq:new_expectations}
\end{equation}
Plugging equations \eqref{eq:new_expectations} into the objective function of optimization problem \eqref{eq:lqr_noise}, collecting like terms and optimizing over system responses as opposed to state and input trajectories, we may rewrite the standard LQR problem as
\begin{equation}
\begin{array}{rl}
\min_{R(t),M(t)} & \sum_{\tau=1}^T (T+1-\tau)\left[\Tr R_\tau^\tp Q R_\tau + \Tr M_\tau^\tp S M_\tau\right] + \sum_{s=1}^{T+1} \Tr R_s^\tp Q_F R_s \\
\st & R_{k+1} = AR_k + BM_k, \ R_1 = I.
\end{array}
\end{equation}

\subsection{Nonzero IC, no noise}
Here we consider the finite-horizon LQR problem with nonzero initial condition and no noise:
\begin{equation}
\begin{array}{rl}
\min_{x(t),u(t)} & \sum_{k=1}^T \left[x_k^\tp Q x_k + u_k^\tp Su_k\right]+ x_{T+1}^\tp Q_F x_{T+1}\\
\st & x_{k+1} = Ax_k + Bu_k, \ x_0 = \xi.
\end{array}
\label{eq:lqr_IC}
\end{equation}
In this case, we can define the disturbance signal driving the system to be 
\begin{equation}
w(t) := \xi \delta_k = (\xi,0,0,\dots).
\end{equation}
It then follows that
\begin{equation}
\begin{array}{rclclcl}
x_k &=& \sum_{t=1}^k R_t w_{k-t} &=& \sum_{t=1}^k R_t \xi \delta_{k-t} &=& R_k \xi \\
u_k &=& \sum_{t=1}^k M_t w_{k-t} &=& \sum_{t=1}^k M_t \xi \delta_{k-t} &=& M_k \xi,
\end{array}
\end{equation}
meaning that the the LQR problem \eqref{eq:lqr_IC} can be rewritten as
\begin{equation}
\begin{array}{rl}
\min_{R(t),M(t)} & \sum_{k=1}^T\left[\Tr R_k^\tp Q R_k\Xi + \Tr M_k^\tp S M_k\Xi\right] +  \Tr R_{T+1}^\tp Q_F R_{T+1}\Xi \\
\st & R_{k+1} = AR_k + BM_k, \ R_1 = I,
\end{array}
\end{equation}
where $\Xi := \xi \xi^\tp$.

\subsection{Putting them together}
We now consider the standard LQR problem with nonzero initial condition and driving process noise
\begin{equation}
\begin{array}{rl}
\min_{x(t),u(t)} & \sum_{k=1}^T \E\left[x_k^\tp Qx_k + u_k^\tp Su_k\right]+ \E\left[x_{T+1}^\tp Q_F x_{T+1}\right]\\
\st & x_{k+1} = Ax_k + Bu_k + w_k, \ x_0 = \xi,
\end{array}
\label{eq:lqr_standard}
\end{equation}
for $w_k {\iid}\mathcal{N}(0,I)$.  Exploiting linearity and the fact that $\E\left[ w_k\xi^\tp\right] = 0$ for all $k$ we can simply combine the results of the previous subsections, allowing us to rewrite the standard LQR problem \eqref{eq:lqr_standard} as
\begin{equation}
\begin{array}{rl}
\min_{R(t),M(t)} & J_\mathrm{stoch}(R(t),M(t)) + J_\mathrm{det}(R(t),M(t),\Xi) \\
\st & R_{k+1} = AR_k + BM_k, \ R_1 = I,
\end{array}
\label{eq:sls_LQR}
\end{equation}
where 
\begin{equation}
\begin{array}{rcl}
J_\mathrm{stoch}(R(t),M(t)) &:=& \sum_{\tau=1}^T (T+1-\tau)\left[\Tr R_\tau^\tp Q R_\tau + \Tr M_\tau^\tp S M_\tau\right] + \sum_{s=1}^{T+1} \Tr R_s^\tp Q_F R_s \\
J_\mathrm{det}(R(t),M(t),\Xi) &:=& \sum_{k=1}^T\left[\Tr R_k^\tp Q R_k\Xi + \Tr M_k^\tp S M_k\Xi\right] +  \Tr R_{T+1}^\tp Q_F R_{T+1}\Xi.
\end{array}
\end{equation}

\section{A redo of the above using block-Toeplitz operators}
In the following we work with the signals and block-Toeplitz lower-triangular operators, and assume $x_0 = 0$, but allow $w_0 \neq 0$\footnote{This is wlog, as $x_1 = w_0$ can be taken as the nonzero initial condition.}:

\begin{equation}
\tf x = \begin{bmatrix} x_1 \\ x_2 \\ \vdots \\ x_{T+1} \end{bmatrix} \ \tf u = \begin{bmatrix} u_1 \\ u_2 \\ \vdots \\ u_{T+1} \end{bmatrix} \ \tf w = \begin{bmatrix} w_0 \\ w_1 \\ \vdots \\ w_T\end{bmatrix} \ \tf R = \begin{bmatrix} R_1 & & & \\ R_2 & R_1 & & \\ \vdots & \ddots & \ddots & \\  R_{T+1} & \cdots & R_2 & R_1 \end{bmatrix} \ \tf M = \begin{bmatrix} M_1 & & & \\ M_2 & M_1 & & \\ \vdots & \ddots & \ddots & \\  M_{T+1} & \cdots & M_2 & M_1 \end{bmatrix} \label{eq:definitions}.
\end{equation}

Further let $Z$ be the block-downshift operator, i.e., a matrix with identity matrices along its first block sub-diagonal and zeros elsewhere.  We can then equivalently write equation \eqref{eq:responses} as
\begin{equation}
\begin{array}{rcl}
\tf x &=& \tf R \tf w \\
\tf u &=& \tf M \tf w.
\label{eq:toep_resp}
\end{array}
\end{equation}

Following the same reasoning as above, it is straightforward to show that the responses $\{\tf R, \tf M\}$ must satisfy the equation
\begin{equation}
\begin{bmatrix} I - Z\A & -Z\B \end{bmatrix}\begin{bmatrix} \tf R \\ \tf M \end{bmatrix} = I,
\label{eq:subspace}
\end{equation}
where
\begin{equation}
\A := \mathrm{blkdiag}\left(A_1,A_2,\dots,A_T\right), \ \B := \mathrm{blkdiag}\left(B_1,B_2,\dots,B_T\right),
\end{equation}

\begin{theorem}
For the dynamics \eqref{eq:dynamics} with block-Toeplitz lower-triangular state feedback law $\tf K$ defining the control action as $\tf u = \tf K \tf x$, the following are true
\begin{enumerate}
\item the affine subspace defined by \eqref{eq:subspace} parameterizes all possible responses \eqref{eq:toep_resp} of the system.
\item for any block-Toeplitz matrices $\{\tf R, \tf M\}$ satisfying \eqref{eq:subspace}, the controller $\tf K = \tf M \tf R^{-1}$ achieves the desired response.
\end{enumerate}
\label{thm:param}
\end{theorem}
\begin{proof}
First note that we can equivalently write the dynamics \eqref{eq:dynamics} as
\begin{equation}
\tf x = Z\A \tf x+ Z\B \tf u + \tf w
\end{equation}
Proof of 1.: Let $\tf K$ be any block-Toeplitz lower-triangular operator, and $\tf u = \tf K \tf x$.  Then
\begin{equation}
\tf x = Z(\A+\B\tf K)\tf x + \tf w,
\end{equation}
which implies that
\begin{equation}
\begin{array}{rcl}
\tf x &=& (I-Z(\A+\B\tf K))^{-1}\tf w \\
\tf u &=& \tf K(I-Z(\A+\B\tf K))^{-1}\tf w.
\end{array}
\end{equation}
It is then easily seen that
\begin{equation}
\begin{bmatrix} I - Z\A \ -Z\B\end{bmatrix}\begin{bmatrix}(I-Z(\A+\B\tf K))^{-1} \\ \tf K (I-Z(\A+\B\tf K))^{-1} \end{bmatrix} = (I - Z\A -Z\B\tf K)(I-Z(\A+\B\tf K))^{-1} = I.
\end{equation}
Proof of 2.: Let $\tf K = \tf M \tf R^{-1}$.  Then
\begin{equation}
\tf x = (I-Z(\A+\B\tf M \tf R^{-1}))^{-1}\tf w.
\end{equation}
But we then have that
\begin{equation}
(I-Z(\A+\B\tf M \tf R^{-1}))^{-1} = ((I-Z\A)\tf R - Z\B \tf M)\tf R^{-1})^{-1} = \tf R ((I-Z\A)\tf R - Z\B \tf M)^{-1}  = \tf R
\end{equation}
where the last equality follows form the fact that $\{\tf R, \tf M\}$ satisfy \eqref{eq:subspace}.  Similarly we have that
\begin{equation}
\tf u = \tf M \tf R^{-1}\tf x = \tf M \tf R^{-1} \tf R \tf w = \tf M \tf w,
\end{equation}
where the second equality follows from the fact that $\tf x = \tf R \tf w$.
\end{proof}

Next we consider a robust variant of the above theorem.  Before stating the result, we let
\begin{equation}
\tf \Delta = \begin{bmatrix} \Delta_1 & & & \\ \Delta_2 & \Delta_1 & & \\ \vdots & \ddots & \ddots & \\  \Delta_{T+1} & \cdots & \Delta_2 & \Delta_1 \end{bmatrix}
\label{eq:delta}
\end{equation}
be an arbitrary block-Toeplitz lower-triangular matrix.

\begin{theorem}
Let $\tf \Delta$ be defined as in equation \eqref{eq:delta}, and suppose that  $\{\tf R, \tf M\}$ satisfy
\begin{equation}
\begin{bmatrix}I - Z\A & -Z\B \end{bmatrix}\begin{bmatrix} \tf R \\ \tf M \end{bmatrix} = I + \tf \Delta.
\label{eq:robust}
\end{equation}

If $(I+\Delta_1)^{-1}$ exists, then the controller $\tf K = \tf M \tf R^{-1}$ achieves the system response
\begin{equation}
\begin{bmatrix}
\tf R \\ \tf M \end{bmatrix} (I + \tf \Delta)^{-1}
\label{eq:robust_response}
\end{equation}
\end{theorem}
\begin{proof}
If $(I+\Delta_1)^{-1}$ exists, then so does $(I + \tf \Delta)^{-1}$ does too, and therefore the constraint \eqref{eq:robust} is equivalent to
\begin{equation}
\begin{bmatrix}I - Z\A & -Z\B \end{bmatrix}\begin{bmatrix} \tf R \\ \tf M \end{bmatrix}(I + \tf \Delta)^{-1} = I.
\end{equation}

Then noting that $\tf K = \tf M \tf R^{-1} = \tf M (I + \tf \Delta)^{-1} (\tf R (I + \tf \Delta)^{-1})^{-1} $, we then have, by Theorem \ref{thm:param}, that the controller $\tf K =  \tf M \tf R^{-1} $ achieves the system responses \eqref{eq:robust_response}.
\end{proof}

\section{LQR optimal control problems}
Going forward, we assume that $\tf w \iid \mathcal{N}(0,I)$.  For simplicity of notation, we assume that $u_{T+1}$ is included in the cost functional, i.e., that the cost we seek to minimize is
\begin{equation}
\begin{array}{rl}
\min_{x(t),u(t)} & \sum_{k=1}^{T+1} \E\left[x_k^\tp Q_k x_k + u_k^\tp S_ku_k\right] \\
\st & x_{k+1} = Ax_k + Bu_k + w_k, \ x_0 = 0.
\end{array}
\label{eq:lqr_toep}
\end{equation}

This problem can be rewritten as
\begin{equation}
\begin{array}{rl}
\min_{\tf R,\tf M} & \bignorm{\begin{bmatrix} \Qq^{\frac{1}{2}} & 0 \\ 0 & \Ss^{\frac{1}{2}}\end{bmatrix}\begin{bmatrix}\tf R \\ \tf M\end{bmatrix}}_F^2 \\
\st & \begin{bmatrix} I - Z\A & -Z\B \end{bmatrix}\begin{bmatrix} \tf R \\ \tf M \end{bmatrix} = I,
\end{array}
\label{eq:lqr_compact}
\end{equation}
where
\begin{equation}
\Qq := \mathrm{blkdiag}(Q_1,Q_2, \dots, Q_{T+1}), \ \Ss := \mathrm{blkdiag}(S_1,S_2, \dots, S_{T+1}).
\end{equation}

\section{A statement of the Learning-LQR result}
The above conditions/results should look extremely similar to what we have seen in the infinite horizon setting.  Letting $\Ahh = \A + \DA$ and $\Bhh = \B + \DB$, with the assumption that $\|\DA\| \leq \epsilon_A$ and $\|\DB\|\leq \epsilon_B$, we can step through the same proof procedure as that in the infinite horizon setting, replacing $\hinf$ norms with operator norms, and $\mathcal{H}_2$ norms with Frobenius norms, to obtain an analogous result.  

We define the finite-time resolvent $\Res{\tf M}$ of an operator $\tf M$ to be
\begin{equation}
\Res{\tf M} := (I - Z\tf M)^{-1},
\end{equation}
and let 
\begin{equation}
\tf \Delta := - Z(\DA + \DB \tf K_\star)\Res{\A + \B\tf K_\star},
\end{equation}
for $\tf K_\star$ the optimal LQR controller.  Further let 
\begin{equation}
\zeta := (\epsilon_A + \epsilon_B \| \tf K_\star\|)\Res{\A + \B\tf K_\star}.
\end{equation}
Then, if $\zeta \leq \frac{1}{5}$, the controller $\tf K=\tf M \tf R^{-1}$ computed by solving the optimization problem
\begin{equation}
\begin{array}{rl}
\min_{\gamma \in [0,1)} \frac{1}{1-\gamma} \min_{\tf R, \tf M} & \bignorm{\begin{bmatrix} \Qq^{\frac{1}{2}} & 0 \\ 0 & \Ss^{\frac{1}{2}}\end{bmatrix}\begin{bmatrix}\tf R \\ \tf M\end{bmatrix}}_F \\
\st & \begin{bmatrix} I - Z\Ahh & -Z\Bhh \end{bmatrix}\begin{bmatrix} \tf R \\ \tf M \end{bmatrix} = I, \ \sqrt{2}\bignorm{\begin{matrix} \epsilon_A\tf R \\ \epsilon_B\tf M\end{matrix}} \leq \gamma
\end{array}
\end{equation}
achieves the relative performance bound
\begin{equation}
\frac{J(\A,\B,\tf K) - J_\star}{J_\star} \leq 5 \zeta,
\end{equation}
where $J_\star$ is the optimal performance achievable if $\A$ and $\B$ are known.

\end{document}
%%% Local Variables:
%%% mode: latex
%%% TeX-master: t
%%% End:

